\documentclass[12pt]{article}
\usepackage{geometry}
\usepackage{graphicx}
\usepackage{float}
\usepackage{listings}

\lstset{frame=tb,
  language=bash,
  aboveskip=3mm,
  belowskip=3mm,
  showstringspaces=false,
  columns=flexible,
  basicstyle={\small\ttfamily},
  numbers=none,
  numberstyle=\tiny\color{gray},
  keywordstyle=\color{blue},
  commentstyle=\color{dkgreen},
  stringstyle=\color{mauve},
  breaklines=true,
  breakatwhitespace=true,
  tabsize=3
}
\begin{document}

\begin{flushright}
140024255\\
CS3105\\
Practical 2, Neural Nets
\end{flushright}

This practical is involved in testing whether the Rapidly Exploring Random Tree or the Potential Fields robot algorithms work better for path finding in various situations. This involves testing both of the systems in free space as well as with a multitude of obstacle courses.

\section{20 Questions}
\paragraph*{Pre-Processing and Input}

\paragraph*{Hidden Layering}

\paragraph*{Output and Post-Processing}

\paragraph*{Training}

\paragraph*{Choosing a Question}

\paragraph*{When to Guess}

\paragraph*{Answers to Question}
%What are the similarities and differences betweeen the way your system generalizes and standard neural generalization

\paragraph*{I/O}

\paragraph*{Extension}
%Evaluate how well your system keeps the number of hidden units under control. Improve your system to keep the number of hidden units under better control.

\paragraph*{Documentation}



\section{Self Programming}
\paragraph*{Pre-Processing and Input}

\paragraph*{Hidden Layering}

\paragraph*{Output and Post-Processing}

\paragraph*{Training}

\paragraph*{Internal States}

\paragraph*{Answers to Question}
%How does the system's final internal state structure compare with a minimal manually created structure?
%How fault tolerant is your system?
%What further design features would be required for your self-programming system to (a) learn rules such as addition using the carry rule (b) replace conventional manual programming?

\paragraph*{I/O}

\paragraph*{Extension}

\paragraph*{Documentation}

%figures go here
%\begin{figure}
%\centering
%\includegraphics[width=350]{.jpg}
%\caption{The pic.}
%\end{figure}


\end{document}

